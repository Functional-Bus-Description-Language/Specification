\chapter{Overview}

\section{Scope}

This document specifies the syntax and semantics of the Functional Bus Description Language (FBDL).

\section{Purpose}

This document is intended for the implementers of tools supporting the language and for users of the language.
The focus is on defining the valid language constructs, their meanings and implications for the hardware and software that is specified or configured, how compliant tools are required to behave, and how to use the language.

\section{Word usage}

The terms “required”, “shall”, “shall not”, “should”, “should not”, “recommended”, “may”, and “optional” in this document are to be interpreted as described in the IETF Best Practices Document 14, RFC 2119.1

\section{Syntactic description}

The formal syntax of the FBDL is described by means of context-free syntax using a simple variant of the Backus-Naur Form (BNF).
In particular:
\begin{enumerate}[label=\alph*)]
	\item Boldface words are used to denote keywords, for example:\newline \textbf{mask}\newline Keywords shall be used only in those places indicated by the syntax.
\end{enumerate}
